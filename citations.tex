\documentclass[oneside,12pt]{memoir} % Font size
\usepackage{adjustbox}
\usepackage{fancyvrb}
\newcommand{\bbfamily}{\fontencoding{U}\fontfamily{bbold}\selectfont}
\DeclareMathAlphabet{\mathbbold}{U}{bbold}{m}{n}

\newenvironment{myverb}{%
 \VerbatimEnvironment
 \begin{adjustbox}{max width=\linewidth}
 \begin{BVerbatim}
  }{
  \end{BVerbatim}
 \end{adjustbox}
}

\input{structure.tex} % Include the file that specifies the document structure and layout

\title{%
	Python:\\
	Early Objects\\
	A GNU Approach} % Book title
%\subtitle{A GNU Approach}
	
\author{Chris Trenary} % Author
\newcommand{\edition}{Manuscript} % Book edition
%\maketitles
%----------------------------------------------------------------------------------------

%----------------------------------------------------------------------------------------
%	HEADER AND FOOTER FORMATS
%----------------------------------------------------------------------------------------

\begin{document}
\chapter{Citations}

%Lambda Calculus
 Turing, A. M. (December 1937). "Computability and \ensuremath{\lambda}-Definability". \emph{The Journal of Symbolic Logic}. \textbf{2} (4): 153–163. JSTOR 2268280. doi:10.2307/2268280. \textsuperscript{1}\\
 
  Church, A. (1932). "A set of postulates for the foundation of logic". \emph{Annals of Mathematics}. Series 2. \textbf{33} (2): 346–366. JSTOR 1968337. doi:10.2307/1968337\textsuperscript{2}
  
  Church, A. (1940). "A Formulation of the Simple Theory of Types". Journal of Symbolic Logic. 5 (2): 56–68. JSTOR 2266170. doi:10.2307/2266170.
  
 
\end{document}