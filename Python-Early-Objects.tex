

%%%%%%%%%%%%%%%%%%%%%%%%%%%%%%%%%%%%%%%%%
% eBook 
% LaTeX Template
% Version 1.0 (29/12/14)
%
% This template has been downloaded from:
% http://www.LaTeXTemplates.com
%
% Original author:
% Luis Cobo (luiscobogutierrez@gmail.com) with extensive modifications by:
% Vel (vel@latextemplates.com)
% with further modification by:
% Chris (trenary.chrys@gmail.com) 
%
% License:
% CC BY-NC-SA 3.0 (http://creativecommons.org/licenses/by-nc-sa/3.0/)
%
%%%%%%%%%%%%%%%%%%%%%%%%%%%%%%%%%%%%%%%%%

%----------------------------------------------------------------------------------------
%	DOCUMENT CONFIGURATIONS AND INFORMATION
%----------------------------------------------------------------------------------------

\documentclass[oneside,12pt]{memoir} % Font size
\usepackage{adjustbox}
\usepackage{fancyvrb}
\newcommand{\bbfamily}{\fontencoding{U}\fontfamily{bbold}\selectfont}
\DeclareMathAlphabet{\mathbbold}{U}{bbold}{m}{n}

\newenvironment{myverb}{%
 \VerbatimEnvironment
 \begin{adjustbox}{max width=\linewidth}
 \begin{BVerbatim}
  }{
  \end{BVerbatim}
 \end{adjustbox}
}

%%%%%%%%%%%%%%%%%%%%%%%%%%%%%%%%%%%%%%%%%
% eBook
% Structural Definitions File
% Version 1.0 (29/12/14)
%
% Created by:
% Vel (vel@latextemplates.com)
% With minor modification by:
% Chris (trenary.chrys@gmail.com)
% 
% This file has been downloaded from:
% http://www.LaTeXTemplates.com
%
% License:
% CC BY-NC-SA 3.0 (http://creativecommons.org/licenses/by-nc-sa/3.0/)
%
%%%%%%%%%%%%%%%%%%%%%%%%%%%%%%%%%%%%%%%%%

%----------------------------------------------------------------------------------------
%	REQUIRED PACKAGES
%----------------------------------------------------------------------------------------

\usepackage[utf8]{inputenc} % Required for inputting international characters
\usepackage[T1]{fontenc} % Output font encoding for international characters

\usepackage[osf]{libertine} % Use the Libertine font
\usepackage{microtype} % Improves character and word spacing

\usepackage{tikz} % Required for drawing custom shapes
\definecolor[named]{color01}{rgb}{.2,.4,.6} % Color used in the title page
\usepackage{wallpaper} % Required for setting background images (title page)

\usepackage[unicode=true,bookmarks=true,bookmarksnumbered=false,bookmarksopen=false,breaklinks=false,pdfborder={0 0 1},backref=section,colorlinks=false]{hyperref} % PDF meta-information specification
\newcommand{\shortTitle}{Python: Early Objects}

%----------------------------------------------------------------------------------------
%	PAPER, MARGIN AND HEADER/FOOTER SIZES
%----------------------------------------------------------------------------------------

\setstocksize{12.5cm}{9.4cm} % Paper size
\settrimmedsize{\stockheight}{\stockwidth}{*} % No trims
\setlrmarginsandblock{18pt}{18pt}{*} % Left/right margins
\setulmarginsandblock{36pt}{36pt}{*} % Top/bottom margins
\setheadfoot{14pt}{12pt} % Header/footer height
\setheaderspaces{*}{8pt}{*} % Extra header space

%----------------------------------------------------------------------------------------
%	FOOTNOTE CUSTOMIZATION
%----------------------------------------------------------------------------------------

\renewcommand{\foottextfont}{\itshape\footnotesize} % Font settings for footnotes
\setlength{\footmarkwidth}{.1em} % Space between the footnote number and the text
\setlength{\footmarksep}{.1em} % Space between multiple footnotes on the same page
\renewcommand*{\footnoterule}{} % Remove the rule above the first footnote
\setlength{\skip\footins}{2\onelineskip} % Space between the body text and the footnote

%----------------------------------------------------------------------------------------
%	HEADER AND FOOTER FORMATS
%----------------------------------------------------------------------------------------

\makepagestyle{mio} % Define a new custom page style
\setlength{\headwidth}{\textwidth} % Header the same width as the text
\makeheadrule{mio}{\textwidth}{0.1mm} % Header rule height
% Use short title in the header if one is defined, otherwise leave the title out
%$if(short-title)$
\makeoddhead{mio}{\scriptsize{\theauthor\hskip.2cm\vrule\hskip.2cm\itshape{\shortTitle}}}{}{} % Header specification
%$else$
%\makeoddhead{mio}{\scriptsize{\theauthor}}{}{} % Header specification
%$endif$
\makeoddfoot{mio}{}{\scriptsize {\thepage \quad \vrule \quad \thelastpage}}{} % Footer specification

\makeoddfoot{plain}{}{\footnotesize {\thepage \quad \vrule \quad \thelastpage}}{} % Pages of chapters
\pagestyle{mio} % Set the page style to the custom style defined above
%----------------------------------------------------------------------------------------
%	PART FORMAT
%----------------------------------------------------------------------------------------

\renewcommand{\partnamefont}{\centering\sffamily\itshape\Huge} % Part name font specification
\renewcommand{\partnumfont}{\sffamily\Huge} % Part number font specification
\renewcommand{\parttitlefont}{\centering\sffamily\scshape} % Part title font specification
\renewcommand{\beforepartskip}{\null\vskip.618\textheight} % Whitespace above the part heading

%----------------------------------------------------------------------------------------
%	CHAPTER FORMAT
%----------------------------------------------------------------------------------------

\makechapterstyle{Tufte}{ % Define a new chapter style
\renewcommand{\chapterheadstart}{\null \vskip3.5\onelineskip} % Whitespace before the chapter starts
\renewcommand{\printchaptername}{\large\itshape\chaptername} % "Chapter" text font specification
\renewcommand{\printchapternum}{\LARGE\thechapter \\} % Chapter number font specification
\renewcommand{\afterchapternum}{} % Space between the chapter number and text
\renewcommand{\printchaptertitle}[1]{ % Chapter title font specification
\raggedright
\itshape\Huge{##1}}
\renewcommand{\afterchaptertitle}{
\vskip3.5\onelineskip
}}
\chapterstyle{Tufte} % Set the chapter style to the custom style defined above

%----------------------------------------------------------------------------------------
%	SECTION FORMAT
%----------------------------------------------------------------------------------------

\setsecheadstyle{\sethangfrom{\noindent ##1}\raggedright\sffamily\itshape\Large} % Section title font specification
\setbeforesecskip{-.6\onelineskip} % Whitespace before the section
\setaftersecskip{.3\onelineskip} % Whitespace after the section

%----------------------------------------------------------------------------------------
%	SUBSECTION FORMAT
%----------------------------------------------------------------------------------------

\setsubsecheadstyle{\sethangfrom{\noindent  ##1}\raggedright\sffamily\large\itshape} % Subsection title font specification
\setbeforesubsecskip{-.5\onelineskip} % Whitespace before the subsection
\setaftersubsecskip{.2\onelineskip} % Whitespace after the subsection

%----------------------------------------------------------------------------------------
%	SUBSUBSECTION FORMAT
%----------------------------------------------------------------------------------------

\setsubsubsecheadstyle{\sethangfrom{\noindent ##1}\raggedright\sffamily\itshape} % Subsubsection title font specification
\setbeforesubsubsecskip{-.5\onelineskip} % Whitespace before the subsubsection
\setaftersubsubsecskip{.1\onelineskip} % Whitespace after the subsubsection

%----------------------------------------------------------------------------------------
%	CAPTION FORMAT
%----------------------------------------------------------------------------------------

\captiontitlefont{\itshape\footnotesize} % Caption font specification
\captionnamefont{\footnotesize} % "Caption" text font specification

%----------------------------------------------------------------------------------------
%	QUOTATION ENVIRONMENT FORMAT
%----------------------------------------------------------------------------------------

\renewenvironment{quotation}
{\par\leftskip=1em\vskip.5\onelineskip\em}
{\par\vskip.5\onelineskip}

%----------------------------------------------------------------------------------------
%	QUOTE ENVIRONMENT FORMAT
%----------------------------------------------------------------------------------------

\renewenvironment{quote}
{\list{}{\em\leftmargin=1em}\item[]}{\endlist\relax}

%----------------------------------------------------------------------------------------
%	MISCELLANEOUS DOCUMENT SPECIFICATIONS
%----------------------------------------------------------------------------------------

\setlength{\parindent}{1em} % Paragraph indentation

\midsloppy % Fewer overfull lines - used in the memoir class and allows a setting somewhere between \fussy and \sloppy

\checkandfixthelayout % Tell memoir to implement the above % Include the file that specifies the document structure and layout

\title{%
	Python:\\
	Early Objects\\
	A GNU Approach} % Book title
%\subtitle{A GNU Approach}
	
\author{Chris Trenary} % Author
\newcommand{\edition}{Manuscript} % Book edition
%\maketitle
%----------------------------------------------------------------------------------------

%----------------------------------------------------------------------------------------
%	HEADER AND FOOTER FORMATS
%----------------------------------------------------------------------------------------

\begin{document}
	

%----------------------------------------------------------------------------------------
%	TITLE PAGE
%----------------------------------------------------------------------------------------

\thispagestyle{empty} % Suppress page numbering
%\ThisCenterWallPaper{1.12}{littlered.jpg} % Add the background image, the first argument is the scaling - adjust this as necessary so the image fits the entire page

\begin{tikzpicture}[remember picture,overlay]
\node [rectangle, rounded corners, fill=white, opacity=0.75, anchor=south west, minimum width=3cm, minimum height=6cm] (box) at (-0.5,-10) (box){}; % White rectangle - "minimum width/height" adjust the width and height of the box; "(-0.5,-10)" adjusts the position on the page
\node[anchor=west, color01, xshift=-1.5cm, yshift=-0.4cm, text width=2.9cm, font=\sffamily\scriptsize] at (box.north){\edition}; % "Text width" adjusts the wrapping width, "xshift/yshift" adjust the position relative to the white rectangle
\node[anchor=west, color01, xshift=-1.5cm, yshift=-2cm, text width=5cm, font=\sffamily\bfseries\scshape\Large] at (box.north){\thetitle}; % "Text width" adjusts the wrapping width, "xshift/yshift" adjust the position relative to the white rectangle
\node[anchor=west, color01, xshift=-1.5cm, yshift=-5cm, text width=2.9cm, font=\sffamily\bfseries] at (box.north){\theauthor}; % "Text width" adjusts the wrapping width, "xshift/yshift" adjust the position relative to the white rectangle
\end{tikzpicture}

\newpage % Make sure the following content is on a new page
%----------------------------------------------------------------------------------------
%	LICENSE
%----------------------------------------------------------------------------------------

    Copyright (C)  2016  Chris Trenary.
    Permission is granted to copy, distribute and/or modify this document
    under the terms of the GNU Free Documentation License, Version 1.3
    or any later version published by the Free Software Foundation;
    with no Invariant Sections, no Front-Cover Texts, and no Back-Cover Texts.
    A copy of the license is included in the section entitled "GNU
    Free Documentation License".

%\LLCornerWallPaper{1.12}{gfdl-logo-small.png}

\newpage % Make sure the following content is on a new page
%----------------------------------------------------------------------------------------
%	TABLE OF CONTENTS
%----------------------------------------------------------------------------------------

\tableofcontents % Prints the table of contents

%----------------------------------------------------------------------------------------
%	INTRODUCTION SECTION
%----------------------------------------------------------------------------------------

\chapter*{Introduction} % Introduction chapter suppressed from the table of contents

\begin{quote}
If the writer of these lines has succeeded in providing some material for clarifying these problems, he may regard his labours as not having been fruitless.\\
--V. Lenin
\end{quote}

I will spare you, my dear reader, the Introduction prattle that you have undoubtedly read countless times before––go ahead and skip to the mission statement. If you have yet to read a proper IT or programming tome, I will attempt to provide the best substitute that a neonate such as myself can conjure. 
\\

Throughout my years of working with technology––from hardware, Information Technology, System Administration, and finally Software Engineering and my first forays into Machine Learing––I have read many books. As do most Technomancers*. As we have all paid our dues studying the type-set characters within tomes of Xeroxed paper or amalgamations of ones and zeroes translated into a fascimile of type-set characters upon a (possibly) back-lit screen, we have learned what works and what does not. While reference manual-esque direct and technical format is very useful to the learned, seasoned, and worldly adminstrator, architect, developer, what-have-you, that time-honoured and much beloved format does not work for most for purposes of instruction or tutorial. Especially for those nascent programmers whom seek wisdom from their peers and forebears at the young hours of the morning, addled with caffeine, and desperately trying to coax a wonder of some calibre from arcane formulae. 

\footnote{* Or Tech Priests, depending upon your persuasion and standing with the Cult of Mars}.

\large Mission Statement: \\
 The objectives of this fledgling text: \\

1. To help the user learn how to think from an Object-oriented perspective. \\
2. To impart some practical knowledge of the Python programming language. \\
3. To (hopefully) impart some of the author's love for programming and abstract mathematics. \\
\\



\chapter*{A brief note on Python} % Introduction chapter suppressed from the table of contents

	It should be noted that Python is an immensely high level language and actually has nothing to do with reptiles. But the creator, \emph{Guido van Rossum}, most definitely named it for the BBC series \emph{Monty Python's Flying Circus}. \\
	
	This text is also focusing on Python 3.6; I will be covering the proper syntax for 3.6. Depending on feedback, I may include a section that reviews the differences between Python3 (3.x) and Python (2.x). 
	
\section*{Thanks}

Here, I would like to thank friends and instructors I have had over the years that have added to my education and grounding as both a Computer Scientist, Mathematician, and programmer.\\

Marcia Elliott, Kristian Lodden, Michael Clingan-Siverly, Sikder Rezwanul Huq, Andy (I need to find your last name!), Dr. Michelle Ruse, Dr. Douglas Jones, Constantine Craft, Joe Hetrick, Dr. Geb Thomas, Professor Joseph Struss, Dr. Alberto Maria Segre, Dr. Aaron Stump, Dr. Mo Liang, Dr. Denise Szecsei, Dr. Sriram Pemmaraju, Dr. Sukumar Ghosh, Dr. Inez Curto, 




%----------------------------------------------------------------------------------------
%	BOOK PART
%----------------------------------------------------------------------------------------
\chapter*{A Rose by any other Name}
\section{A (some what) brief Introduction to Set Theory}
\label{set_theory}
\section{An (Informal) introduction to Formal Logic}
\label{formal_logic}

\section{A (very) Brief History of Proof}
\label{proof}

\part{%
The Fairy Tales	\\
\large Or How I Learned to Stop \\
		Worrying and Love the Object}

%----------------------------------------------------------------------------------------
%	CHAPTER ONE
%----------------------------------------------------------------------------------------

\chapter{%
Little Red Riding Hood \\
\large First Steps}

% Python Chevrons:
% {>}{>}{>} = >>>
\section{(Parse)l Tongue}
\label{parsel_tongue}

As tradition dictates, the very first program we should learn, is "Hello, World!"

Given Python's console interface:\\

{>}{>}{>} print('Hello, World!')\\

Should yield:\\
Hello, World!\\
To screen, unless something horrible has happened.\\


Now, what exactly did we just do? \\
"print()", as I'm confident that you had figured out already, is the mechanism that printed "Hello, World!" to the screen. 


Execute:\\
{>}{>}{>} print("4")\\
{>}{>}{>} print(4)\\

So, are the 4's different? Luckily for us, Python has a nifty tool to let us know. Go ahead and give these a try: \\
{>}{>}{>} type("4")\\
{>}{>}{>} type(4)\\


It appears that they are different indeed. <type 'str'> and ,<type 'int'> are -very- different. Type str refers to string, a datatype that is often used to display and parse text. Type int, refers to integer, a numeric datatype that is used to represent numbers, of the discrete variety. \\

The String (str) datatype:\\

Type Str in Python:\\


The Integer (int) datatype:\\

Useful for discrete mathematics and counting things that don't or should not have fractions, barring dark humour. Discrete meaning countable, in the reasonable sense. \\

Type Int in Python:\\
Is much more flexible with its peer datatypes in Python than some of the other mainstream programming languages. Python has no quarrel with operations between ints, floats and numeric conversion between operations. \\



\subsection{Numeric Operations} % (fold)
\label{sub:numeric_operations}

% subsection subsection_name (end)Numeric Operations:\\
%Numeric Operations:\\

Let's us see a few examples of what can be done, hopefully without having to return any parrots. \\
The following examples are used in Terminal, console, however REPL works for you. I will denote lines of exposition with an octothorpe(\#).\\

{>}{>}{>} 3 + 2\\
6\\
{>}{>}{>} 2*42-9\\
75\\
{>}{>}{>} (30-15)*2/4\\
7.5\\

\# Here, we received a fractional answer. Despite the default numeric type for Python being int. In Python 2.x, the answer would have been 7 ala true integer/floor division.\\
\# In Python 3.x, classic division ex: 4/2 will \textbf{\emph{always}} return a \emph{float}.\\
{>}{>}{>} 4/2\\
2.0\\

\# In Python 3.x, we needs must specify floor division.\\
{>}{>}{>} (30-15)*2//4\\
7\\


As you can see, Python 3.6 follows the formal Order of Operations.\\
Parentheses\\
Exponent\\
Multiplication\\
Division\\
Addition\\
Subtraction\\

The questionable mnemonic device that many Midwesterners learned growing up was:\\

'Please Excuse My Dear Aunt Sally'\\

One often wonders what dear Sally did to reach such notoriety, mayhaps she was an avid Spam enthusiast?\\
{>}{>}{>}124/(2*42-9)\\
1.6533333333\\

\# In Python 2.x, this performs integer division, resulting in 1 and not the floating point witnessed above. This is also called a comment, when lines of text follow an octothorpe. \\



\subsection {A Note on Comments}
\label{comments}

%A Note on Comments:\\
\# Are very important to your life and sanity as a programmer. And those that will work with you and work on code you write. Their importance can not be over stated.\\

But that doesn't quite describe 'what' a comment is. \\
\# A succinct version of the "Hello, World!" application.\\

print('Hello, World!') \\


The comment above did exactly what comments are supposed to do. It succinctly described the operation of the program. In Python, the octothorpe (\#) denotes a the beginning of a \emph{single line comment}.\\

\subsubsection{Multiline Comments}
\label{sub:sub:multiline_comments}

In order to perform \emph{multi-line commenting}, a different trick must be used. \emph{Triple quoted strings} is the closest approximation. That being said, these are originally used as \emph{Docstrings } (the first line in a function, class, or module). Whenever \emph{Triple quoted strings} are used outside of the first line in a function, class, or module, they are ignored, making this a feasible alternative to Octothorping every line you wish to comment. Guido van Rossum stated this as a "Pro tip".\\

That being said, the style guide for Python, the PEP8, puts an emphasis on concurrent \emph{single-line} commenting.\\

PEP8 suggested \emph{multi-line} comments:\\
\#\\
\# This is\\
\# a\\
\# Multi-line comment\\
\#\\

"Pro tip" \emph{multi-line} comments:\\
{'}{'}{'}\\
This is a Multi-line comment\\
{'}{'}{'}\\


The use and format of comments will take many forms depending on requirements set by coworkers, professors, managers, and the like. One should be ready to alter their format at the drop of a hat.\\

The commenting formats used herein will be both descriptive and documentation. With an official release, I will likely adhere to PEP8, with the exception of the "pro tip". \\

 It should be noted that I will strive to offer comments to any and all code snippets I include to foster this notion. \\

\section{Math Operations, Pi-thon}
\label{math_operations}

A few words on Discrete:\\
Putting a definition to Discrete numbers is important for the coming discussion and is a central theme to the logic behind the friendly integer. \\

If some one where to ask a normal humanbeing how many numbers are between 1 and 5, they would likely say three, '2,3,4'. Where as a Mathematician or Engineer would be wont to say infinity. Both are simultaneously true. Discrete refers to 'countable' events, items, positions, whatever else you can conceive. From the perspective of Discrete counting, there are only three numbers between 1 and 5. \\
On the other hand, continuous numbers, are just that––continuous. Between 1 and 2, there is 1.5, or 1 and 1/2. There is also 1.25, 1.75, 1.99, 1.0000000000000001, and far too many more for me to reasonably list in linear time. These numbers are continuous and belong to the world of fractions, both positive and negative infinities. \\

The int is a good data type to start with regarding numbers, a numeric. Good, because it is simple and doesn't quite jump into the extra baggage associated with floats, doubles, and the like (more on them later).\\


\subsection{General Numeric Operations}
\label{sub:general_numeric_operations}

As noted above, here are some more general numeric operations with 'interpreter' examples.\\

{>}{>}{>} 6 + 6\\
12\\
{>}{>}{>} 32-9*2\\
14\\
{>}{>}{>} (32-9*2)/7\\
2.0 

More examples of Python3's use of the Order of Operations and use of Division. Note the use of Division (/) and how the division of the ints resulted in a float. This is a stark contrast from most programming languages and needs must be addressed.\\

As addressed before, should you want the operation to return an 'int', not a float, it can be converted after the Order of Operations has executed with the int(x) operation.\\

{>}{>}{>} int((32-9*2)/7)\\
2\\

/, referred to as \emph{classic division}, will always return a float. 
{>}{>}{>} 21/5\\
4.2\\
{>}{>}{>} 36/6\\
6.0\\

//, \emph{floor division} will perform the similar (not the same) operations as \emph{classic division} and return the answer as an int. \\

{>}{>}{>}21//5\\
4\\
{>}{>}{>}(32-9*2)//7\\
2\\
{>}{>}{>} 36//6\\
6\\
{>}{>}{>} 21//5\\
4\\

\%, \emph{Modulus Division} Will also perform division, after a fashion. Instead of keeping the whole number and disregarding the fractional part, only the fractional part is returned––as its \emph{floor value}. This sees significant use in certain computations, such as rendering. \\
\# \emph{Classic Division}\\
{>}{>}{>}87/9\\
9.666666666\\
\# \emph{Modulus Divison}\\
{>}{>}{>} 87\%9\\
6\\
{>}{>}{>} 13\%4\\
1\\
{>}{>}{>} 36\%6\\
0\\
{>}{>}{>} 120\%7\\
1\\


\textbf{Regarding floor values}:\\
\textbf{Notice} how the outputs, of the \emph{Modulus} operations yield a conclusive and not so fractional answer? Take a look at \ensuremath{87\%9}:\\
Shouldn't this be returned as 666666(ad infinum)? While the remainder is taken as \textbf{x} many beyond the whole number result from \emph{Classic Division}, this does not take into consideration the fractional parts beyond the tenths place. Go ahead and try the following in your interpreter:\\

\# More \emph{Modulus Divison}\\
{>}{>}{>} 86\%9\\
{>}{>}{>} 87\%9\\
{>}{>}{>} 88\%9\\
{>}{>}{>} 89\%9\\
{>}{>}{>} 90\%9\\
{>}{>}{>} 91\%9\\
{>}{>}{>} 8\%9\\

Notice anything interesting? The number gets larger and larger until the modulus equals 9––at which point––the modulus is 0. 90 modulus 9 yields 0 as 90 / 9 is 10, no remainder. 91 modulus 9 is 1, as the remainder is 1. That is to say that 91 is greater than 90 by 1, ergo the remainder is 1. \\

Now, the enigma here is 8 modulus 9. We get 8 as our remainder. It is 8 greater than 0. 0 mod 9 is 0. Some more experimentation might be in order....\\

\# Once more, with feeling!\\

{>}{>}{>} 9\%9\\
{>}{>}{>} 0\%9\\
{>}{>}{>} 1\%9\\
{>}{>}{>} 2\%9\\
{>}{>}{>} 3\%9\\
{>}{>}{>} 4\%9\\
{>}{>}{>} 5\%9\\
{>}{>}{>} 6\%9\\
{>}{>}{>} 7\%9\\
{>}{>}{>} 8\%9\\
{>}{>}{>} -1\%9\\
{>}{>}{>} -2\%9\\
{>}{>}{>} -9\%9\\
{>}{>}{>} -10\%9\\
{>}{>}{>} 10 \%9\\

Isn't this interesting? Approaching negative infinity, a different trend arises, one that mirrors the one approaching positive infinity. 1\%9 is 1, -1\%9 is 8. As is -10\%9. -11 mod 9 is 7. The remainder is yet a positive value as it counts the absolute value between whole number divisions.\\

Notice that the modulus increases from -9\%9 until 0\%9? Then this trend continues upward from 1 mod 9 \= 1, with the modulus increasing until 9\%9 which is then 0 with 10 mod 9 being 1.

\section{Numerical Operations}
\label{numerical_operations}


%Table Example

\begin{table}[h!]

  \centering
  \caption{Operations of the Numeric types.}
  \label{tab:numerical\ operations}
  \begin{adjustbox}{width=1\textwidth}
  \begin{tabular}{l|c}
	\hline
    Operation & Description\\
  
	\hline
	
	\ensuremath{p + q}& Sums p and q\\
    \hline
    \ensuremath{p - q}& Subtracts q from p\\
	\hline
	\ensuremath{p / q}& Divides p by q\\
	\hline
	\ensuremath{p * q}& Multiplies p by q\\
	\hline
	\ensuremath{p // q}& Floor divides p by q\\
	\hline
	\ensuremath{p \% q}& Modulus of p and q\\
	\hline
	\ensuremath{abs(p)}& The magnitude or absolute value of p\\
	\hline
	\ensuremath{-p}& Negation of p\\
	\hline
	\ensuremath{+p}& p\\
	\hline
	\ensuremath{int(p)}& Converts p into an int\\
	\hline
	\ensuremath{p + q}& Converts p into a float\\
	\hline
	\ensuremath{complex(r, i)}& Converts a real number, \emph{r},\\
	%\hline
								&
	with an imaginary part, \emph{i}. By default \emph{i} is 0  \\
	\hline
	\ensuremath{c.conjugate()}& 
Converts a complex number, \emph{c},\\
								&
	into the conjugate of \emph{c}\\
	\hline
	\ensuremath{divmod(p,q)}&
The pair of operations,\ensuremath{(p//q,p\%q)} \\
								&
p floor division y q and p modulus by q\\
	\hline
\ensuremath{pow(p,q)}	&
p raised to the power of q\\
\ensuremath{p**q}		&
						\\
\hline

	
	
  \end{tabular}
  \end{adjustbox}
\end{table}
%\caption{A caption after the table}


\section{String Operations}
\label{string_operations}





\section{The String Class}
\label{the_string_class}

\begin{table}[h!]

  \centering
  \caption{Operations of the String types.}
  \label{tab:String\ operations}
  \begin{adjustbox}{width=1\textwidth}
  \begin{tabular}{l|c}
	\hline
    Operation & Description\\
  
	\hline
	
	\ensuremath{p + q}& Sums p and q\\
    \hline
    \ensuremath{p - q}& Subtracts q from p\\
	\hline
	\ensuremath{p / q}& Divides p by q\\
	\hline
	\ensuremath{p * q}& Multiplies p by q\\
	\hline
	\ensuremath{p // q}& Floor divides p by q\\
	\hline
	\ensuremath{p \% q}& Modulus of p and q\\
	\hline
	\ensuremath{abs(p)}& The magnitude or absolute value of p\\
	\hline
	\ensuremath{-p}& Negation of p\\
	\hline
	\ensuremath{+p}& p\\
	\hline
	\ensuremath{int(p)}& Converts p into an int\\
	\hline
	\ensuremath{p + q}& Converts p into a float\\
	\hline
	\ensuremath{complex(r, i)}& Converts a real number, \emph{r},\\
	%\hline
								&
	with an imaginary part, \emph{i}. By default \emph{i} is 0  \\
	\hline
	\ensuremath{c.conjugate()}& 
Converts a complex number, \emph{c},\\
								&
	into the conjugate of \emph{c}\\
	\hline
	\ensuremath{divmod(p,q)}&
The pair of operations,\ensuremath{(p//q,p\%q)} \\
								&
p floor division y q and p modulus by q\\
	\hline
\ensuremath{pow(p,q)}	&
p raised to the power of q\\
\ensuremath{p**q}		&
						\\
\hline

	
	
  \end{tabular}
  \end{adjustbox}
\end{table}

\section{\textbf{The Philosopher's Stone}\\
And then there were Objects}
\label{object_oriented_programming_introduction}
\section{\textbf{A Wolf in Sheep's Clothing}\\
 Introduction to Polymorphism}
 \label{polymorphism}

%----------------------------------------------------------------------------------------
%	CHAPTER TWO
%----------------------------------------------------------------------------------------

\chapter{Hansel and Gretel \\
\large Defeating the Witch}
\section{\textbf{Enter the Function}}
\label{functions_introduction}
\section{\textbf{Sanctum Sempra}}
\label{funcitons_methods}


%----------------------------------------------------------------------------------------
%	CHAPTER THREE
%----------------------------------------------------------------------------------------

\chapter{Rupunzel \\
\large Gordian's Knot}

\section{\textbf{\ensuremath{\lambda} Calculus}}
\label{lambda_calculus}

% Example using Blackboard bold outside of math mode
%\ensuremath{\mathbbold{R}}\\


	
\emph{\ensuremath{\lambda}p.p+1}\\

\textbf{Within} the \emph{Lambda Calculus}, the above is a valid definition of the variable p increment function. But what are the implications for programming? Below, we assign something to \emph{p}, lets us say we make 41 the argument for the function. Also, note what's missing? There is no name to the proper Lambda function defined above. Lambda functions do not possess names as they are anonymous operations.\\


\emph{(\ensuremath{\lambda}p.p+1)41}\\

This is similar to: \ensuremath{\lambda}p.p+1, if it were a function with 41 being an argument passed to the function. As a whole, this is known as a \emph{Lambda Expression}.\\

\[(\lambda p.p+1)41 \Rightarrow 41+1 \Rightarrow 42.\]\\



\emph{Lambda Calculus} is a very relevant topic for both Computer Scientists and programmers. This hearkens back to the time of Computer Science in it's infancy and maintains it's importance to this day. The Lambda Calculus was originally gifted to us by Alonzo Church \footnote{Church, A(1932). "A set of postulates for the foundation of logic". \emph{Annals of Mathematics}. Series 2. \textbf{33} (2): 346–366. JSTOR 1968337. doi:10.2307/1968337} while he furthered his research into the \emph{foundations of Mathematics}. At later dates, he also gave us \emph{Untyped Lambda Calculus}\footnote{ Church, A. (1936). "An unsolvable problem of elementary number theory". American Journal of Mathematics. 58 (2): 345–363. JSTOR 2371045. doi:10.2307/2371045.} for computations and \emph{Simply Typed Lambda Calculus} that was published in the \emph{Journal of Symbolic Logic}\footnote{Church, A. (1940). "A Formulation of the Simple Theory of Types". Journal of Symbolic Logic. 5 (2): 56–68. JSTOR 2266170. doi:10.2307/2266170.}.\\

The Lambda Calculus utilizes distinct expressions, rules, and operations unique to itself. Among these operations exist rules that allow the manipulation of variables and expressions.\\

These expressions are known as \emph{Lambda Terms} and they are defined as valid terms if they follow proper syntax, not unlike any other programming language. \\
There are three guidelines that determine whether something is or is not a valid \emph{Lambda Term}:\\

1. Variables, such as q, are valid Lambda terms. \\
2. if p is a lambda term, and q is a variable, then (\ensuremath{\lambda}q.p) is a Lambda term (here it is known as a \emph{Lambda abstraction})\\
3. if p and j are lambda terms, then (pj) is a lambda term (known as an \emph{application})\\

If something does not follow these rules, then it is simply not a lambda term. 

\subsection{\textbf{The Syntax of Lambda Calculus}}
\label{sub:the_syntax_of_lambda_calculus}





\textbf{The} core syntax of Lambda Expressions:\\

\textbf{I. All variables are lambda expressions.}\\
\textbf{II. If P and Q are lambda expressions}
 \ensuremath{\Rightarrow}
\textbf{so are 1, 2, and 3 below.}\\
\textbf{1.(P)}\\
\textbf{2.\ensuremath{\lambda}zy.P}\\
\textbf{3. PQ}\\


\textbf{1.} Clearly states that ()(parenthesis) are legal nesting syntax, this is nifty because this keeps our notation consistent without breaking any rules.\\
\textbf{2.} This is a not so formal definition for an \emph{abstraction}: function with a parameter of zy with a body of \textbf{P}.\\
\textbf{3.} This is a Lambda \emph{applicaiton}: We perform \textbf{P} upon \textbf{Q}\\

\section{\textbf{Lambda Functions in Python}}
\label{lambda_functions_in_python}

%https://stackoverflow.com/questions/890128/why-are-python-lambdas-useful

Lambda expressions in Python are generally of the form:\\
% TODO
% This is brokem and fills page, need to fix
% I think this is fixed

\begin{myverb}
	
lambda_expr::= "lambda" [param_list]: expression

\end{myverb}

Expressions defined in this way have the behaviour of classically defined functions with returns.\\
Like so:\\
def <lambda>(args):\\
	return expression\\
Similarities in \emph{function}:\\ 

\begin{myverb}
	
lambda_expr::= lambda q: q+3

\end{myverb}

Or Literally in the interpreter: \\
	
{>}{>}{>} l = lambda q: q+3\\
{>}{>}{>}l(7)\\
10\\

Which is the same expected behaviour of:\\
{>}{>}{>}p=function(i):\\
...			return i + 3\\
...\\
{>}{>}{>}p(7)\\
10\\


%----------------------------------------------------------------------------------------

\part{%
Illiad's Tales of Troy\\
 The Manticore and other Abhors}
 
This is the customary section whence we face the problems of Computer Science and bear witness to the imaginative genious of Computer Scientists and Mathematicians, the unknown, the unsolved, and the unsolvable. 

\chapter{%
 Regarding Discrete Mathematics
}
\textbf{%
\large This} chapter serves as both a soft introduction and refresher to some of the relevant Discrete Mathematics in programming. It is my hope, that I am capable of communicating some of these important concepts for any level of programmer. 

\section{Counting}
\label{counting}
\subsection{Probability}
\label{sub:probability}
\section{Sets}
\section{Summation}
\section{Matrices}
\section{Graphs}
\label{graphs}
\chapter{Problems in Computer Science}
\section{Notable Solved Problems}
\label{notable_solved_problems}
\subsection{Detecting a Loop in a Linked List}
\label{detecting_a_loop_in_a_linked_list}
\chapter{Intractable Problems}
%----------------------------------------------------------------------------------------
\part{%
GFDL\\
GNU Free Documentation License}




GNU Free Documentation License\\

Version 1.3, 3 November 2008\\

Copyright © 2000, 2001, 2002, 2007, 2008 Free Software Foundation, Inc. <http://fsf.org/>\\

Everyone is permitted to copy and distribute verbatim copies of this license document, but changing it is not allowed.\\

0. PREAMBLE\\

The purpose of this License is to make a manual, textbook, or other functional and useful document "free" in the sense of freedom: to assure everyone the effective freedom to copy and redistribute it, with or without modifying it, either commercially or noncommercially. Secondarily, this License preserves for the author and publisher a way to get credit for their work, while not being considered responsible for modifications made by others.

This License is a kind of "copyleft", which means that derivative works of the document must themselves be free in the same sense. It complements the GNU General Public License, which is a copyleft license designed for free software.

We have designed this License in order to use it for manuals for free software, because free software needs free documentation: a free program should come with manuals providing the same freedoms that the software does. But this License is not limited to software manuals; it can be used for any textual work, regardless of subject matter or whether it is published as a printed book. We recommend this License principally for works whose purpose is instruction or reference.

1. APPLICABILITY AND DEFINITIONS\\

This License applies to any manual or other work, in any medium, that contains a notice placed by the copyright holder saying it can be distributed under the terms of this License. Such a notice grants a world-wide, royalty-free license, unlimited in duration, to use that work under the conditions stated herein. The "Document", below, refers to any such manual or work. Any member of the public is a licensee, and is addressed as "you". You accept the license if you copy, modify or distribute the work in a way requiring permission under copyright law.

A "Modified Version" of the Document means any work containing the Document or a portion of it, either copied verbatim, or with modifications and/or translated into another language.

A "Secondary Section" is a named appendix or a front-matter section of the Document that deals exclusively with the relationship of the publishers or authors of the Document to the Document's overall subject (or to related matters) and contains nothing that could fall directly within that overall subject. (Thus, if the Document is in part a textbook of mathematics, a Secondary Section may not explain any mathematics.) The relationship could be a matter of historical connection with the subject or with related matters, or of legal, commercial, philosophical, ethical or political position regarding them.

The "Invariant Sections" are certain Secondary Sections whose titles are designated, as being those of Invariant Sections, in the notice that says that the Document is released under this License. If a section does not fit the above definition of Secondary then it is not allowed to be designated as Invariant. The Document may contain zero Invariant Sections. If the Document does not identify any Invariant Sections then there are none.

The "Cover Texts" are certain short passages of text that are listed, as Front-Cover Texts or Back-Cover Texts, in the notice that says that the Document is released under this License. A Front-Cover Text may be at most 5 words, and a Back-Cover Text may be at most 25 words.

A "Transparent" copy of the Document means a machine-readable copy, represented in a format whose specification is available to the general public, that is suitable for revising the document straightforwardly with generic text editors or (for images composed of pixels) generic paint programs or (for drawings) some widely available drawing editor, and that is suitable for input to text formatters or for automatic translation to a variety of formats suitable for input to text formatters. A copy made in an otherwise Transparent file format whose markup, or absence of markup, has been arranged to thwart or discourage subsequent modification by readers is not Transparent. An image format is not Transparent if used for any substantial amount of text. A copy that is not "Transparent" is called "Opaque".

Examples of suitable formats for Transparent copies include plain ASCII without markup, Texinfo input format, LaTeX input format, SGML or XML using a publicly available DTD, and standard-conforming simple HTML, PostScript or PDF designed for human modification. Examples of transparent image formats include PNG, XCF and JPG. Opaque formats include proprietary formats that can be read and edited only by proprietary word processors, SGML or XML for which the DTD and/or processing tools are not generally available, and the machine-generated HTML, PostScript or PDF produced by some word processors for output purposes only.

The "Title Page" means, for a printed book, the title page itself, plus such following pages as are needed to hold, legibly, the material this License requires to appear in the title page. For works in formats which do not have any title page as such, "Title Page" means the text near the most prominent appearance of the work's title, preceding the beginning of the body of the text.

The "publisher" means any person or entity that distributes copies of the Document to the public.

A section "Entitled XYZ" means a named subunit of the Document whose title either is precisely XYZ or contains XYZ in parentheses following text that translates XYZ in another language. (Here XYZ stands for a specific section name mentioned below, such as "Acknowledgements", "Dedications", "Endorsements", or "History".) To "Preserve the Title" of such a section when you modify the Document means that it remains a section "Entitled XYZ" according to this definition.

The Document may include Warranty Disclaimers next to the notice which states that this License applies to the Document. These Warranty Disclaimers are considered to be included by reference in this License, but only as regards disclaiming warranties: any other implication that these Warranty Disclaimers may have is void and has no effect on the meaning of this License.

2. VERBATIM COPYING\\

You may copy and distribute the Document in any medium, either commercially or noncommercially, provided that this License, the copyright notices, and the license notice saying this License applies to the Document are reproduced in all copies, and that you add no other conditions whatsoever to those of this License. You may not use technical measures to obstruct or control the reading or further copying of the copies you make or distribute. However, you may accept compensation in exchange for copies. If you distribute a large enough number of copies you must also follow the conditions in section 3.

You may also lend copies, under the same conditions stated above, and you may publicly display copies.

3. COPYING IN QUANTITY\\

If you publish printed copies (or copies in media that commonly have printed covers) of the Document, numbering more than 100, and the Document's license notice requires Cover Texts, you must enclose the copies in covers that carry, clearly and legibly, all these Cover Texts: Front-Cover Texts on the front cover, and Back-Cover Texts on the back cover. Both covers must also clearly and legibly identify you as the publisher of these copies. The front cover must present the full title with all words of the title equally prominent and visible. You may add other material on the covers in addition. Copying with changes limited to the covers, as long as they preserve the title of the Document and satisfy these conditions, can be treated as verbatim copying in other respects.

If the required texts for either cover are too voluminous to fit legibly, you should put the first ones listed (as many as fit reasonably) on the actual cover, and continue the rest onto adjacent pages.

If you publish or distribute Opaque copies of the Document numbering more than 100, you must either include a machine-readable Transparent copy along with each Opaque copy, or state in or with each Opaque copy a computer-network location from which the general network-using public has access to download using public-standard network protocols a complete Transparent copy of the Document, free of added material. If you use the latter option, you must take reasonably prudent steps, when you begin distribution of Opaque copies in quantity, to ensure that this Transparent copy will remain thus accessible at the stated location until at least one year after the last time you distribute an Opaque copy (directly or through your agents or retailers) of that edition to the public.

It is requested, but not required, that you contact the authors of the Document well before redistributing any large number of copies, to give them a chance to provide you with an updated version of the Document.

4. MODIFICATIONS\\

You may copy and distribute a Modified Version of the Document under the conditions of sections 2 and 3 above, provided that you release the Modified Version under precisely this License, with the Modified Version filling the role of the Document, thus licensing distribution and modification of the Modified Version to whoever possesses a copy of it. In addition, you must do these things in the Modified Version:

A. Use in the Title Page (and on the covers, if any) a title distinct from that of the Document, and from those of previous versions (which should, if there were any, be listed in the History section of the Document). You may use the same title as a previous version if the original publisher of that version gives permission.
B. List on the Title Page, as authors, one or more persons or entities responsible for authorship of the modifications in the Modified Version, together with at least five of the principal authors of the Document (all of its principal authors, if it has fewer than five), unless they release you from this requirement.
C. State on the Title page the name of the publisher of the Modified Version, as the publisher.
D. Preserve all the copyright notices of the Document.
E. Add an appropriate copyright notice for your modifications adjacent to the other copyright notices.
F. Include, immediately after the copyright notices, a license notice giving the public permission to use the Modified Version under the terms of this License, in the form shown in the Addendum below.
G. Preserve in that license notice the full lists of Invariant Sections and required Cover Texts given in the Document's license notice.
H. Include an unaltered copy of this License.
I. Preserve the section Entitled "History", Preserve its Title, and add to it an item stating at least the title, year, new authors, and publisher of the Modified Version as given on the Title Page. If there is no section Entitled "History" in the Document, create one stating the title, year, authors, and publisher of the Document as given on its Title Page, then add an item describing the Modified Version as stated in the previous sentence.
J. Preserve the network location, if any, given in the Document for public access to a Transparent copy of the Document, and likewise the network locations given in the Document for previous versions it was based on. These may be placed in the "History" section. You may omit a network location for a work that was published at least four years before the Document itself, or if the original publisher of the version it refers to gives permission.
K. For any section Entitled "Acknowledgements" or "Dedications", Preserve the Title of the section, and preserve in the section all the substance and tone of each of the contributor acknowledgements and/or dedications given therein.
L. Preserve all the Invariant Sections of the Document, unaltered in their text and in their titles. Section numbers or the equivalent are not considered part of the section titles.
M. Delete any section Entitled "Endorsements". Such a section may not be included in the Modified Version.
N. Do not retitle any existing section to be Entitled "Endorsements" or to conflict in title with any Invariant Section.
O. Preserve any Warranty Disclaimers.
If the Modified Version includes new front-matter sections or appendices that qualify as Secondary Sections and contain no material copied from the Document, you may at your option designate some or all of these sections as invariant. To do this, add their titles to the list of Invariant Sections in the Modified Version's license notice. These titles must be distinct from any other section titles.

You may add a section Entitled "Endorsements", provided it contains nothing but endorsements of your Modified Version by various parties—for example, statements of peer review or that the text has been approved by an organization as the authoritative definition of a standard.

You may add a passage of up to five words as a Front-Cover Text, and a passage of up to 25 words as a Back-Cover Text, to the end of the list of Cover Texts in the Modified Version. Only one passage of Front-Cover Text and one of Back-Cover Text may be added by (or through arrangements made by) any one entity. If the Document already includes a cover text for the same cover, previously added by you or by arrangement made by the same entity you are acting on behalf of, you may not add another; but you may replace the old one, on explicit permission from the previous publisher that added the old one.

The author(s) and publisher(s) of the Document do not by this License give permission to use their names for publicity for or to assert or imply endorsement of any Modified Version.

5. COMBINING DOCUMENTS\\

You may combine the Document with other documents released under this License, under the terms defined in section 4 above for modified versions, provided that you include in the combination all of the Invariant Sections of all of the original documents, unmodified, and list them all as Invariant Sections of your combined work in its license notice, and that you preserve all their Warranty Disclaimers.

The combined work need only contain one copy of this License, and multiple identical Invariant Sections may be replaced with a single copy. If there are multiple Invariant Sections with the same name but different contents, make the title of each such section unique by adding at the end of it, in parentheses, the name of the original author or publisher of that section if known, or else a unique number. Make the same adjustment to the section titles in the list of Invariant Sections in the license notice of the combined work.

In the combination, you must combine any sections Entitled "History" in the various original documents, forming one section Entitled "History"; likewise combine any sections Entitled "Acknowledgements", and any sections Entitled "Dedications". You must delete all sections Entitled "Endorsements".

6. COLLECTIONS OF DOCUMENTS\\

You may make a collection consisting of the Document and other documents released under this License, and replace the individual copies of this License in the various documents with a single copy that is included in the collection, provided that you follow the rules of this License for verbatim copying of each of the documents in all other respects.

You may extract a single document from such a collection, and distribute it individually under this License, provided you insert a copy of this License into the extracted document, and follow this License in all other respects regarding verbatim copying of that document.

7. AGGREGATION WITH INDEPENDENT WORKS\\

A compilation of the Document or its derivatives with other separate and independent documents or works, in or on a volume of a storage or distribution medium, is called an "aggregate" if the copyright resulting from the compilation is not used to limit the legal rights of the compilation's users beyond what the individual works permit. When the Document is included in an aggregate, this License does not apply to the other works in the aggregate which are not themselves derivative works of the Document.

If the Cover Text requirement of section 3 is applicable to these copies of the Document, then if the Document is less than one half of the entire aggregate, the Document's Cover Texts may be placed on covers that bracket the Document within the aggregate, or the electronic equivalent of covers if the Document is in electronic form. Otherwise they must appear on printed covers that bracket the whole aggregate.

8. TRANSLATION\\

Translation is considered a kind of modification, so you may distribute translations of the Document under the terms of section 4. Replacing Invariant Sections with translations requires special permission from their copyright holders, but you may include translations of some or all Invariant Sections in addition to the original versions of these Invariant Sections. You may include a translation of this License, and all the license notices in the Document, and any Warranty Disclaimers, provided that you also include the original English version of this License and the original versions of those notices and disclaimers. In case of a disagreement between the translation and the original version of this License or a notice or disclaimer, the original version will prevail.

If a section in the Document is Entitled "Acknowledgements", "Dedications", or "History", the requirement (section 4) to Preserve its Title (section 1) will typically require changing the actual title.

9. TERMINATION\\

You may not copy, modify, sublicense, or distribute the Document except as expressly provided under this License. Any attempt otherwise to copy, modify, sublicense, or distribute it is void, and will automatically terminate your rights under this License.

However, if you cease all violation of this License, then your license from a particular copyright holder is reinstated (a) provisionally, unless and until the copyright holder explicitly and finally terminates your license, and (b) permanently, if the copyright holder fails to notify you of the violation by some reasonable means prior to 60 days after the cessation.

Moreover, your license from a particular copyright holder is reinstated permanently if the copyright holder notifies you of the violation by some reasonable means, this is the first time you have received notice of violation of this License (for any work) from that copyright holder, and you cure the violation prior to 30 days after your receipt of the notice.

Termination of your rights under this section does not terminate the licenses of parties who have received copies or rights from you under this License. If your rights have been terminated and not permanently reinstated, receipt of a copy of some or all of the same material does not give you any rights to use it.

10. FUTURE REVISIONS OF THIS LICENSE\\

The Free Software Foundation may publish new, revised versions of the GNU Free Documentation License from time to time. Such new versions will be similar in spirit to the present version, but may differ in detail to address new problems or concerns. See http://www.gnu.org/copyleft/.

Each version of the License is given a distinguishing version number. If the Document specifies that a particular numbered version of this License "or any later version" applies to it, you have the option of following the terms and conditions either of that specified version or of any later version that has been published (not as a draft) by the Free Software Foundation. If the Document does not specify a version number of this License, you may choose any version ever published (not as a draft) by the Free Software Foundation. If the Document specifies that a proxy can decide which future versions of this License can be used, that proxy's public statement of acceptance of a version permanently authorizes you to choose that version for the Document.

11. RELICENSING\\

"Massive Multiauthor Collaboration Site" (or "MMC Site") means any World Wide Web server that publishes copyrightable works and also provides prominent facilities for anybody to edit those works. A public wiki that anybody can edit is an example of such a server. A "Massive Multiauthor Collaboration" (or "MMC") contained in the site means any set of copyrightable works thus published on the MMC site.

"CC-BY-SA" means the Creative Commons Attribution-Share Alike 3.0 license published by Creative Commons Corporation, a not-for-profit corporation with a principal place of business in San Francisco, California, as well as future copyleft versions of that license published by that same organization.

"Incorporate" means to publish or republish a Document, in whole or in part, as part of another Document.

An MMC is "eligible for relicensing" if it is licensed under this License, and if all works that were first published under this License somewhere other than this MMC, and subsequently incorporated in whole or in part into the MMC, (1) had no cover texts or invariant sections, and (2) were thus incorporated prior to November 1, 2008.

The operator of an MMC Site may republish an MMC contained in the site under CC-BY-SA on the same site at any time before August 1, 2009, provided the MMC is eligible for relicensing.


\chapter{Citations}

%Lambda Calculus
 Turing, A. M. (December 1937). "Computability and \ensuremath{\lambda}-Definability". \emph{The Journal of Symbolic Logic}. \textbf{2} (4): 153–163. JSTOR 2268280. doi:10.2307/2268280. \textsuperscript{1}\\
 
  Church, A. (1932). "A set of postulates for the foundation of logic". \emph{Annals of Mathematics}. Series 2. \textbf{33} (2): 346–366. JSTOR 1968337. doi:10.2307/1968337\textsuperscript{2}
  
 
\end{document}













